\section{Lineare Programmierung \skript{49}}

\subsection{Problemformulierung}
  \subsubsection{Normalformen}
    Ein lineares Programm kann in verschiedenen Formen geschrieben werden:
    
    \begin{tabularx}{\textwidth}{|l|X|}
      \hline
      Allgemeine Form & 
          max / min $cx$ \newline
          $a^ix \leq b_i,$ \newline
          $a^ix = b_i,$ \newline
          $a^ix \geq b_i,$ \newline
          $x_j \geq 0,$ \newline
          $x_j \text{ frei},$ \newline
          $x_j \leq 0$
          \\
      \hline
      Kanonische Form & 
          max $cx$ \newline
          $a^ix \leq b_i,$ \newline
          $x_j \geq 0$ \newline
          \em oder \em\newline
          min $cx$ \newline
          $a^ix \geq b_i,$ \newline
          $x_j \geq 0$
         \\
      \hline
      Standardform & 
        max / min $cx$ \newline
        $a^ix = b_i,$ \newline
        $x_j \geq 0$
        \\
      \hline
      Ungleichsform & 
        max $cx$ \newline
        $a^ix \leq b_i,$ \newline
        \em oder \em \newline
        min $cx$ \newline
        $a^ix \geq b_i,$
        \\
      \hline
    \end{tabularx}

    Zusätzlich wird zwischen Maximierungs- und Minimierungsproblemen unterschieden.
    
  \subsubsection{Umformulierungen \skript{51}}
    \begin{aufzaehlung}
      \item $x_j \leq 0 \qquad \rightsquigarrow \qquad x_j := -\bar{x}_j, \quad \bar{x}_j \geq 0$
      \item $x_j \text{ frei} \qquad \rightsquigarrow \qquad x_j := x_j^+ - x_j^-, \quad x_j^+, x_j^- \geq 0$
      \item $a^i x = b_i \qquad \rightsquigarrow \qquad a^i x \leq b_i, \quad a^i x \geq b_i$
      \item $a^i x \leq b_i \qquad \rightsquigarrow \qquad -a^i x \geq -b_i$ bzw.\\
            $a^i x \geq b_i \qquad \rightsquigarrow \qquad -a^i x \leq -b_i$
      \item $a^i x \leq b_i \qquad \rightsquigarrow \qquad a^i x + x_i^s = b_i, \quad x_i^s \geq 0$ bzw.\\
            $a^i x \geq b_i \qquad \rightsquigarrow \qquad a^i x - x_i^s = b_i, \quad x_i^s \geq 0$
    \end{aufzaehlung}
  

\subsection{Simplex Algorithmus}

